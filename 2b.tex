\subsection*{b) Neville's Algorithm}
In this part we want to apply Neville's Algorithm to the 20 given datapoints and then we interpolate 1000 valuesfor the full Lagrange polynomial and plot all 20 points. \\
To calculate the steps inbetween, we use the formula:
\begin{align}
    p_{01}(x) = \frac{(x_1 - x) p_0 + (x-x_0)p_1}{x_1-x_0}
\end{align}
Then we iterate over every interval with $p_i$ the y value to the given $x_i$.\\
\begin{figure}
    \centering
    \includegraphics[scale=0.7]{./neville.jpg}
    \caption{This plot shows the evaluation of the polynomial from neville's algoritm. }
    \label{fig:enter-label}
\end{figure}
One property of neville's algorithm is that the edge terms stay fixed and the values inbetween get interpolated related to the choosen intervals. That is a important difference to the approach in a). 
Out script is given by:
\lstinputlisting{b2.py}

